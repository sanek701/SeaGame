\subsection{Cерверная часть}

Основными функиями сервера в данном проекте является контроль за деятельносться пользователей и передача действий от одного игрока другому. Проанализировав правилы игры можно развернуть диаграмму вариантов использования для сервера, которую в дальшейем раскроем либо с помощью диаграмм активности, последовательностей и состояний, либо с помощью сценариев.
  
\begin{figure}[htp]
\centering
\includegraphics[width=15cm]{images/useserver.png}
\caption{Диаграмма прецедентов сервера}
\label{fig14}
\end{figure}

В диаграмме классов [\ref{fig15}] участвует два класса, играющие роль посредников передачи сообщений клиент-сервер и в данной части рассматриваться не будут(подробнее об этих классах в главе 2). 

\begin{figure}[htp]
\centering
\includegraphics[width=18cm]{images/class_server.png}
\caption{Диаграмма классов сервера}
\label{fig15}
\end{figure}

Класс Ship здесь используется как вспомогательный класс, фактически как дополнительная структура. Класс Game выполняет те функции, которые были описаны выше, то есть проверяет корректность действий и сообщает о действиях оппонента. Именно к классу Game по большей части будут относиться диаграммы, детализирующие варианты использования.

Обратимся к прецеденту <<Установка кораблей>>. На диаграмме [\ref{fig14}] видно, что он сотоит из 3-х частей. Установку и удаление распишем в виде сценариев, а перемещение представим в виде диаграммы состояний [\ref{fig16}].  


\textbf{Создание кораблей:}
	
		\begin{itemize}		
			\item Запрос на создание корабля T от X на клетке [$x,y$];		
			\item Проверить что клетка [$x,y$] свободна;
			\item Проверить, что клетка [$x,y$] на поле игрока X;
			\item Оправить изменения поля обоим игрокам.		
		\end{itemize}


\textbf{Удаление кораблей:}

		\begin{itemize}		
			\item Запрос на удаление корабля T от X на клетке [$x,y$];		
			\item Проверить что клетка [$x,y$] занята кораблем игрока Х;
		\item Отправить изменения поля обоим игрокам.	
		\end{itemize}


\begin{figure}[htp]
\centering
\includegraphics[width=16cm]{images/statemove.png}
\caption{Диаграмма проверки перемещения корабля}
\label{fig16}
\end{figure}

 Следует обратить внимание, что отправка изменений должна осуществляться не только оппоненту, но и тому игроку, который совершал ход, так как все действия игроков не отображаются и не фиксируются в клиент-приложении, а осуществляются по командам сервера. Более подробно это описано на примере диаграммы смены статусов игры[\ref{fig17}], изображенной в виде диаграммы последовательностей или, более общно, в диаграмме[\ref{fig}].

\begin{figure}[htp]
\centering
\includegraphics[width=18cm]{images/srvstate.png}
\caption{Диаграмма смены статусов}
\label{fig17}
\end{figure}

В силу равноправности пользователе(т.е. создателя игры и игрока, который присоединился), диаграмму[\ref{fig17}] можно зеркально отобразить. Этого не было сделано в целях сохранения доступности и понятности схемы. Так же на ней был отмечен статус <<Ход>>, который на самом деле состоит и нескольких подстатусов, таких как: перемещение корабля(является обязательной частью, неподлежащей исключению по правилам), атака корабля противника и, как следствие ответ противника, взрыв атомной бомбы(только один раз за игру). Последние три действия не являются обязательными для завершения хода. Остановимся подробнее на атаке кораблей, тоесть рассмотрим прецедент <<Сравнение>>.Этот вариант может быть связан не только с отдельными короблями, но и с некоторой их совокупностью. Блоком называется объединение двух или трёх кораблей, каждый из которых стоит на соседних клетках по горизонтали или вертикали. Общая схема выполнения сравнения отражена здесь[\ref{fig18}], а о мощности блоков и отдельных кораблей подробно написано в [\ref{roganov}].
\begin{figure}[htp]
\centering
\includegraphics[width=12cm]{images/ask.png}
\caption{Диаграмма сравнения блоков(кораблей)}
\label{fig18}
\end{figure}
\endinput
