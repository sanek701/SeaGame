\section{Введение}
  Разработка современных информационных систем является сложным, многоэтапным процессом. 
  Выделяют несколько основных частей разработки ПО:
  \begin{itemize}
		\item Анализ требований к ПО;
		\item Проектирование архитектуры ПО;
    \item Детальное проектирование;
    \item Кодирование и отладочное тестирование;
    \item Поддержку процесса получения ПО.
  \end{itemize} 
  При проектировании информационной системы удобно использовать язык UML.
  UML (англ. Unified Modeling Language~-- унифицированный язык моделирования)~-- язык графического описания для   
объектного моделирования в области разработки программного обеспечения. UML является языком широкого профиля, это 
открытый стандарт, использующий графические обозначения для создания абстрактной модели системы, называемой 
UML-моделью. UML был создан для определения, визуализации, проектирования и документирования в основном программных систем
  Преимущества UML
  \begin{itemize}
    \item UML объектно-ориентированный, в результате чего методы описания результатов анализа и проектирования семантически близки к методам программирования на современных ОО-языках;
    \item UML позволяет описать систему практически со всех возможных точек зрения и разные аспекты поведения системы;
    \item Диаграммы UML сравнительно просты для чтения после достаточно быстрого ознакомления с его синтаксисом;
    \item UML расширяет и позволяет вводить собственные текстовые и графические стереотипы, что способствует его применению не только в сфере программной инженерии;
    \item UML получил широкое распространение и динамично развивается.
  \end{itemize}
\newpage
\endinput


